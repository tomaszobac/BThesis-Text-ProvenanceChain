%%%%%%%%%%%%%%%%%%%%%%%%%%%%%%%%%%%%%%%%%%%%%%%%%%%%%%%%%%%%%%%%%%%%
%% I, the copyright holder of this work, release this work into the
%% public domain. This applies worldwide. In some countries this may
%% not be legally possible; if so: I grant anyone the right to use
%% this work for any purpose, without any conditions, unless such
%% conditions are required by law.
%%%%%%%%%%%%%%%%%%%%%%%%%%%%%%%%%%%%%%%%%%%%%%%%%%%%%%%%%%%%%%%%%%%%

\documentclass[
  digital,     %% The `digital` option enables the default options for the
               %% digital version of a document. Replace with `printed`
               %% to enable the default options for the printed version
               %% of a document.
%%  color,       %% Uncomment these lines (by removing the %% at the
%%               %% beginning) to use color in the printed version of your
%%               %% document
  oneside,     %% The `oneside` option enables one-sided typesetting,
               %% which is preferred if you are only going to submit a
               %% digital version of your thesis. Replace with `twoside`
               %% for double-sided typesetting if you are planning to
               %% also print your thesis. For double-sided typesetting,
               %% use at least 120 g/m² paper to prevent show-through.
  nosansbold,  %% The `nosansbold` option prevents the use of the
               %% sans-serif type face for bold text. Replace with
               %% `sansbold` to use sans-serif type face for bold text.
  nocolorbold, %% The `nocolorbold` option disables the usage of the
               %% blue color for bold text, instead using black. Replace
               %% with `colorbold` to use blue for bold text.
  lof,         %% The `lof` option prints the List of Figures. Replace
               %% with `nolof` to hide the List of Figures.
  lot,         %% The `lot` option prints the List of Tables. Replace
               %% with `nolot` to hide the List of Tables.
]{fithesis4}
%% The following section sets up the locales used in the thesis.
\usepackage[resetfonts]{cmap} %% We need to load the T2A font encoding
\usepackage[T1,T2A]{fontenc}  %% to use the Cyrillic fonts with Russian texts.
\usepackage[
  main=english, %% By using `czech` or `slovak` as the main locale
                %% instead of `english`, you can typeset the thesis
                %% in either Czech or Slovak, respectively.
  english, german, russian, czech, slovak %% The additional keys allow
]{babel}        %% foreign texts to be typeset as follows:
%%
%%   \begin{otherlanguage}{german}  ... \end{otherlanguage}
%%   \begin{otherlanguage}{russian} ... \end{otherlanguage}
%%   \begin{otherlanguage}{czech}   ... \end{otherlanguage}
%%   \begin{otherlanguage}{slovak}  ... \end{otherlanguage}
%%
%% For non-Latin scripts, it may be necessary to load additional
%% fonts:
\usepackage{paratype}
\def\textrussian#1{{\usefont{T2A}{PTSerif-TLF}{m}{rm}#1}}
%%
%% The following section sets up the metadata of the thesis.
\thesissetup{
    date        = \the\year/\the\month/\the\day,
    university  = mu,
    faculty     = fi,
    type        = bc,
    department  = Department of Computer Systems and Communications,
    author      = Tomáš Zobač,
    gender      = m,
    advisor     = {RNDr. Rudolf Wittner},
    title       = {Implementation of provenance chains traversal},
    TeXtitle    = {Implementation of provenance chains traversal},
    keywords    = {Java, provenance, SOBHA, ÚVT},
    TeXkeywords = {Java, provenance, SOBHA, ÚVT},
    abstract    = {%
      Provenance is a standardized information type that documents the history of an object. It can hold information, such as the origin of an object or previous actions performed on it. This information can be serialized into one of the many supported file formats (e.g., PROVN, XML). These files can then be interconnected, resulting in the creation of a provenance chain. This thesis aims to implement a library for traversing said provenance chains, retrieving information about the precursors or successors of an entity represented in one of the files in the current chain, and optionally retrieving the type of actions performed on the object by the retrieved precursors/successors. The implementation will simulate the operational environment by providing a command-line user interface from where the user can call the mentioned actions on a set of prefactured simulation files. The implementation will follow a W3C PROV-DM standard for provenance notation, and the simulation files will be of a PROVN serialization of provenance data. Additionally to the traversing operations, it will also implement the generation of a provenance metadata to make the simulation of a traversal possible.
    },
    thanks      = {%
      TODO
    },
    bib         = example.bib,
    %% Remove the following line to use the JVS 2018 faculty logo.
    facultyLogo = fithesis-fi,
}
\usepackage{makeidx}      %% The `makeidx` package contains
\makeindex                %% helper commands for index typesetting.
%% These additional packages are used within the document:
\usepackage{paralist} %% Compact list environments
\usepackage{amsmath}  %% Mathematics
\usepackage{amsthm}
\usepackage{amsfonts}
\usepackage{url}      %% Hyperlinks
\usepackage{markdown} %% Lightweight markup
\usepackage{listings} %% Source code highlighting
\lstset{
  basicstyle      = \ttfamily,
  identifierstyle = \color{black},
  keywordstyle    = \color{blue},
  keywordstyle    = {[2]\color{cyan}},
  keywordstyle    = {[3]\color{olive}},
  stringstyle     = \color{teal},
  commentstyle    = \itshape\color{magenta},
  breaklines      = true,
}
\usepackage{floatrow} %% Putting captions above tables
\floatsetup[table]{capposition=top}
\usepackage[babel]{csquotes} %% Context-sensitive quotation marks
\begin{document}
%% The \chapter* command can be used to produce unnumbered chapters:
\chapter*{Introduction}
%% Unlike \chapter, \chapter* does not update the headings and does not
%% enter the chapter to the table of contents. I we want correct
%% headings and a table of contents entry, we must add them manually:
\markright{\textsc{Introduction}}
\addcontentsline{toc}{chapter}{Introduction}

Theses are rumoured to be \enquote{the capstones of education}, so
I decided to write one of my own. If all goes well, I will soon
have a diploma under my belt. Wish me luck!


\chapter{Provenance}
\shorthandoff{-}
Provenance refers to the collective information about the history or origin of something, especially its ownership or location history. It is often used in contexts where it is essential to trace the history of an object, idea, or practice to establish its authenticity, origin, and how it was handled through time. For example, in literature, provenance can relate to the history of a manuscript or a book. Knowing who owned a rare manuscript, where it was kept, and how it was passed down through generations can shed light on its historical importance and authenticity.

In the digital realm, provenance refers to the documentation of the history of data, including its origins, processing, and storage. This is particularly significant, especially in fields such as scientific research, where tracking the origin and modifications of data ensures its reliability and reproducibility. Provenance is the term used to describe all the information and data referred to in the abovementioned examples. It is the collective name given to this type of data.
\shorthandon{-}

\section{PROV-DM} \label{provdm}
\shorthandoff{-}
In order to enable the inter-operable interchange of provenance information, the W3C PROV-DM (Provenance Data Model) standard was created. The PROV-DM represents provenance as a directed graph through nodes and edges. 

Nodes in this graph represent Entities, Activities, and Agents, each playing a distinct role. Entities represent a physical, digital, or other objects central to or associated with the provenance. Activities represent actions and processes conducted on or with these entities. Agents are the nodes that perform activities, effectively creating or influencing entities and activities. Additionally each node has a unique identifier which clearly identifies it in the current provenance graph.

The edges in this graph represent the relations between these nodes. Examples include 'wasDerivedFrom,' 'wasGeneratedBy,' and 'used,' which help map out the model's interactions.

Moreover, the PROV-DM standard includes a concept known as 'Bundles'. These are used to group sets of provenance information, enabling the encapsulation of various model elements. This addition allows for a more nuanced and organized representation of provenance data, enhancing the model's capability to be easily shared and processed. Same as with the nodes in the graph, a Bundle also has a unique identifier, which allows it to be represented in a  for example parent or meta provenance, effectively allowing creation of provenance of a provenance.

The flexibility of PROV-DM is enhanced through its extension support. Users can define new nodes and add custom attributes. This adaptability makes PROV-DM an ideal choice for various applications, from academic research to industry practices.
\shorthandon{-}

\subsection{PROV-N}
\shorthandoff{-}
PROV-N is a vital component of the W3C PROV suite, designed to represent it in a human-readable textual format. This format organizes provenance information into three main sections: namespace declarations, statements, and bundles. 

Namespace declarations simplify the identification of entities, activities, and agents by providing a shorthand prefix to a longer namespace URI. For instance, an entity might be identified using a QualifiedName, such as "ns\_surgery:sampleConnector." This notation combines a namespace URI, a readable prefix, and the local part, ensuring that each component is uniquely identified and easily referenced. Namespaces are declared as follows:

\begin{verbatim}

prefix ns_surgery <ns_surgery_URI>
prefix ns_pathology <ns_pathology_URI>

\end{verbatim}

Statements in PROV-N describe the provenance structure and it's relationships. They can range from simple declarations to complex narratives explaining how different entities interact. For example, an entity might be described as "entity(ns\_surgery:sampleConnector)" with attributes detailing its characteristics:

\begin{verbatim}

entity(ns_surgery:sampleConnector, [
    prov:type=
        'cpm:forwardConnector', 
    cpm:receiverBundleId=
        'ns_pathology:02_scanning.provn', 
    cpm:receiverServiceUri=
        "#URI#"
    ])
    
\end{verbatim}

Bundles, as mentioned in \ref{provdm}, allow for grouping related provenance information. This organization aids in managing complex provenance scenarios, where it is crucial to encapsulate different aspects of the provenance in distinct, manageable units. The finished document, with declared namespaces and a bundle encapsulating its statements, could look like this example:

\begin{verbatim}

document
  prefix ns_surgery <ns_surgery_URI>

  bundle ns_surgery:01_sample_acquisition.provn
    prefix ns_surgery <ns_surgery_URI>
    prefix ns_pathology <ns_pathology_URI>
    prefix cpm <cpm_URI>

    entity(ns_surgery:sampleConnector,[
        prov:type=
            'cpm:senderConnector',
        cpm:receiverBundleId=
            'ns_pathology:02_scanning.provn',
        cpm:receiverServiceUri=
            "#URI#"
    ])
  endBundle
endDocument

\end{verbatim}
\shorthandon{-}

\section{Provenance chain} \label{s-provchain}
\shorthandoff{-}
The provenance chain is a sequence of interconnected Bundles, each encapsulating a graph consisting of a backbone with pre-defined structure and a scenario specific information, which is attached to the backbone in a standardized way. 

Provenance backbone is composed of three main types of Entities designed to establish connections within and between bundles. The first two of the mentioned Entities are backwardConnector and forwardConnector which stand on the endpoints of the backbone. The backwardConnector links to a forwardConnector in the preceding Bundle with same identifier, creating a backward traversal path. Conversely, the forwardConnector establishes a forward link to a backwardConnector in the subsequent Bundle, with the same identifier as well. In the middle of the backbone stands the third Entity called currentConnector. This Entity references the current state of the object described by the provenance, linking it to the current bundle in the chain.

\begin{figure}[htbp]
  \begin{center}
    \includegraphics[width=12.5cm]{fithesis/images/bundleconnection.png}
  \end{center}
  \caption{Connection of two provenance bundles}
  \label{fig:bundleconnection}
\end{figure}

While the example given simplifies the structure, it is important to note that, in practice, a single bundle can contain multiple connectors of each type. This multiplicity allows for connections to more than one bundle on either side of the traversal path, thereby accommodating complex provenance scenarios. The ability to link multiple bundles through various connectors enhances the flexibility and depth of the provenance chain, enabling a more detailed and comprehensive representation of the provenance.

With its series of linked bundles and pre-defined structure, the provenance chain offers a robust framework for documenting, tracking, and analyzing the provenance.
\shorthandon{-}

\section{Additional concepts}
TODO

\section{Real-world distribution}
TODO


\chapter{Design}
\shorthandoff{-}
This chapter delves into the architecture and logic of the library, which was developed as an implementation part of this thesis. The chapter has three main objectives. Firstly, it aims to provide a clear and concise overview of the structure of the library and its components, including their respective roles in the library's lifecycle. Secondly, it discusses the simulation of a real-world distribution and how it works. Finally, it describes the primary logic used for traversing a provenance chain. It is crucial to explain these specific details to understand the technical implementation described in the succeeding chapter. 
\shorthandon{-}

\section{Overview}
TODO

\section{Structure}
\shorthandoff{-}
At its core, the implementation is designed as a modular, multi-compo-\\nent library. Each component serves a distinct role but contributes to cohesive functionality. This modular approach facilitates easy maintenance and scalability. Central to this structure is a layered architecture, segregating the library operations into distinct layers - data retrieval, processing, and presentation. 
The data retrieval layer is responsible for acquiring the PROV-N data, depending on the type of storage, and deserializing it into a unified type. 
The processing layer handles the traversal of the provenance chain as well as the traversal of the directed graph inside each PROV-N document. It is also responsible for evaluating the documents' integrity and retrieving the relevant provenance data.
Finally, the presentation layer provides the interface for user interaction, enabling command inputs and displaying results on the command line.
\shorthandon{-}

\subsection{Components}
\shorthandoff{-}
The library's components are separated into four main parts.
\begin{enumerate}
    \item \textbf{Main UI:}
        The Main UI component is tasked with the initialization of resources for the library and providing a user interface on the command line, where the user can input commands and view the results of the entered queries. The user interface also provides quality-of-life functions, such as command auto-completion, command history, or in-command query history.
    \item \textbf{Configuration:}
        The Configuration component retrieves data from a configuration JSON file and passes it to the library.
    \item \textbf{Tools:}
        The Tools component serves as a toolset for the library with the first set of tools used to retrieve and deserialize PROV-N documents depending on their storage type. 
        The second set of tools is used to resolve persistent identifiers stored in the relevant navigation table.
        The third is used for retrieving hash values from the corresponding meta-provenance, and the last is used for hash creation.
    \item \textbf{Main logic:}
        The Main logic component handles the data processing. It traverses the provenance chain and retrieves the user-specified data. It also assures the correctness of the traversed chain by evaluating the integrity of the document using its hash values stored in the corresponding meta-provenance and comparing it to new hash values created using the last set of tools.
\end{enumerate}
\shorthandon{-}

\section{Simulation}
\shorthandoff{-}
For the implementation to simulate a real-world environment, some regulations had to be established specifically for this thesis. The PROV-N serialization will be the only one used in this thesis. The PROV-N documents used in the implementation were generated and provided by the thesis supervisor using the "cite thesis". These documents follow a predefined structure and naming rules described in the standard currently being developed by the supervisor and his colleagues. The implementation will also include a package for generating meta-provenance, which will be used for the provided documents. Lastly, the resolution of the persistent identifiers will be done with an in-memory navigation table created during the library's initialization process containing all of the provided documents.
\shorthandon{-}

\section{Provenance chain traversal}
\shorthandoff{-}
As mentioned before in section \ref{s-provchain}, the main structure of a bundle consists of three Entities with predefined connector types connected by the wasDerivedFrom statements. As an example of the traversal, assume the algorithm has already traversed one bundle and is now entering a succeeding one with its graph's structure, as shown in Figure.

\begin{figure}[htbp]
  \begin{center}
    \includegraphics[width=9cm]{fithesis/images/examplebigger.png}
  \end{center}
  \caption{Excerpt from a PROV-N document}
  \label{fig:bundleexample}
\end{figure}
\shorthandon{-}

Considering the algorithm arrived from the bundle 01\_sample\_acquisition,
the next step is to use a wasDerivedFrom statement where the
backwardConnector is in the role of a used Entity.

\begin{figure}[htbp]
  \begin{center}
    \includegraphics[width=9cm]{fithesis/images/examplebigger.png}
  \end{center}
  \caption{Excerpt from a PROV-N document}
  \label{fig:bundleexample2}
\end{figure}

After moving to the currentConnector, the algorithm again finds a wasDerivedFrom statement where the currentConnector is in the role of a used Entity, and the forwardConnector is in the role of a generated Entity.

\begin{figure}[htbp]
  \begin{center}
    \includegraphics[width=9cm]{fithesis/images/examplebigger.png}
  \end{center}
  \caption{Excerpt from a PROV-N document}
  \label{fig:bundleexample3}
\end{figure}

After reaching the end of the bundle's graph, the algorithm can enter another bundle by looking into the current forwardConnector's attributes, retrieving the succeeding bundle's identifier (Figure 1.1), and executing the whole process again.

\begin{figure}[htbp]
  \begin{center}
    \includegraphics[width=9cm]{fithesis/images/examplebigger.png}
  \end{center}
  \caption{Excerpt from a PROV-N document}
  \label{fig:bundleexample4}
\end{figure}


\chapter{Implementation}
\shorthandoff{-}
This chapter aims to bridge the gap between the theoretical framework outlined in the Design chapter and the actual realization of the project. It offers an in-depth look at the code structure, libraries used, and the problems that arose during the implementation. Through a clear and systematic presentation, this chapter showcases the application's development process and serves as a valuable resource for understanding the practical challenges and solutions encountered. Reading this chapter will offer insights into the application's architecture and operational mechanics, providing a deeper appreciation of the project's technical complexity.

\section{Project structure}
TODO

\section{Used technologies}
The following part delves into the libraries and tools used in the implementation and the reasoning behind each.
\subsection{ProvToolBox}
ProvToolBox is a library for managing provenance data within the implementation.  
\subsubsection{PROV MODEL}
The 'prov-model' is used across the whole library to create and manipulate provenance documents.
\subsubsection{PROV INTEROP: LIGHT}
Complementary to the prov-model, the 'prov-interop-light' enhances the system's interoperability. This library facilitates the conversion of provenance data into various formats, ensuring that the application can communicate effectively with other provenance-aware systems and tools.
\subsection{Jackson Databind}
Jackson Databind is a library for handling JSON data formats. In ConfigLoader.java and Configuartion.java, 'jackson-databind' parses the configuration file and creates Java objects from JSON. This functionality is crucial in managing configurations within the library.
\subsection{GitLab4J API}
Integrated primarily in GitLabFileLoader.java, the GitLab4J API is a library to facilitate interaction with GitLab's APIs. It automates file retrieval, allowing the library to support git-based provenance storage.
\subsection{JLine Bundle}
JLine Bundle is a library used to enhance the user interface of the implementation, as seen in MainRuntime.java. It provides capabilities such as command-line completion and history, significantly improving the user experience by making the interface more intuitive and responsive.
\subsection{Jansi}
Utilized in MainRuntime.java, Jansi is a library used to add stylistic improvements to the console output. It renders text in different colors and styles, making the console output about the integrity of Bundles more readable and user-friendly.
\subsection{Apache Maven Shade Plugin}
Including the Maven Shade Plugin plays a significant role in the library's build process. This plugin creates an uber-jar, a standalone executable jar file containing all the necessary dependencies. This approach simplifies deployment and distribution, ensuring the library is self-contained and can be run in diverse environments without additional dependency management.


\section*{}
Each library has been chosen and integrated into the implementation to address specific functional requirements and enhance the application's usability, efficiency, and interoperability. The strategic use of these libraries aligns with the broader goal of creating a user-focused, adaptable, and future-proof application.

\section{Runtime}
This section is intended to provide a concise runtime walkthrough of the library. The purpose is to offer insight into how different classes and components interact to produce the desired query results for the user. By following this walkthrough, readers will understand the underlying mechanisms of the library's runtime operation. Starting with the class Main, which serves as a single entry point, calling the static MainRuntime class. MainRuntime serves as the top-level part of the library, statically initializing all resources and simultaneously providing a user interface for command inputs.

For this walkthrough, the queried request is to retrieve the precursors of an Entity by using the 'precursors' command. In the context of the 'precursors' command, i.e., when traversing the provenance chain backward, the mentioned precursors are represented as forwardConnector Entities of the preceding bundles.

After the command is identified, the findPrecursors method is entered with a false argument, as the user does not wish to retrieve the main Activity alongside the precursors. Upon entering the method, the user is prompted to enter the Entity's ID and URI and the ID and URI of the Bundle in which the Entity resides. Next, the correct implementation of the IFileLoader interface is chosen to retrieve the PROV-N file and return it deserialized into a Document object. After that, the integrity of the Document is verified by calling the checkSum method of the Crawler class. With this, the preparation phase ends, and the runtime enters the Crawler's method getPrecursors.

Upon entering, the proper navigational table resolver is chosen, and the connector type of the Entity is retrieved. Next comes the first if branch, checking whether the connector type is a forwardConnector. If not, the method continues; if it is, the current Entity with its Bundle is retrieved as a precursor, and the method continues. Right after the first if branch comes a second one checking whether the connector type is a backwardConnector; if it is, it means the method reached the end of the current Bundle's graph, and it will repeat the same process as in the MainRuntime section, retrieving the following Document and verifying its integrity after which, it calls the getPrecursors method recursively. If not, it will search through the WasDerivedFrom statements in the Document to find which Entity is following in the traversal and call the getPrecursors method recursively with the same Document and the newly found Entity. 

After the runtime is done traversing every Bundle in the chain, it prints those results on the command line, along with confirmation of the verified integrity of each Bundle.

\section{Problems during implementation}
Four main problems were encountered during the implementation due to using the ProvToolBox Java library. The simulation files provided by the supervisor were generated using a Python library, which produced files with some notation differences. The first is that the Python library supports attributes with multiple values using parentheses, while the Java library needs each attribute to have only one value.

\begin{verbatim}

    Python: 
        dct:hasPart=('prefix:local','prefix:local')
    Java:
        dct:hasPart='prefix:local', 
        dct:hasPart='prefix:local' 
        
\end{verbatim}

The second problem came from the Python library supporting the creation of Entities with space in names, which is seen as a syntax error by the Java library.

\begin{verbatim}

    Python:
        entity(prefix:local 1, [prov:type=...
    Java:
        entity(prefix:local_1, [prov:type=... 
        
\end{verbatim}

The third problem was caused because the Python library generates attributes with values encompassed by double quotes. The Java library differentiates between single and double quotes. The single quotes are resolved as the QualifiedName specified inside of them.

\begin{verbatim}

    cpm:receiverBundleId='prefix:local.provn'
    
    Resolved:
        'prefix:{{uri}}local.provn'
        
\end{verbatim}

While the double quotes are regarded as a LangString

\begin{verbatim}

    cpm:receiverServiceUri="#URI#"
    
    Resolved:
        LangString@xxx[value=#URI#,lang=<null>]
        
\end{verbatim}

The last problem came from the fact that the ProvToolBox library, at the time of thesis implementation, was left on version 0.9.3, with the last update being over a year ago, which caused some of the library's features not to work correctly or at all. As of the late summer of 2023, the development seems to have returned, and new updates are being released.
\shorthandon{-}


\chapter{Manual}
\shorthandoff{-}
\section{Set-up}
To utilize the library, it is necessary to have Maven and Java properly configured. To simplify the process for users, the implementation is currently spread across multiple Git repositories, including two on FI and ICS GitLab sites and one on the author's personal GitHub page. To obtain the library, clone the repository or use the download button to retrieve the entire repository bundled in a zip or another archive file.

\subsection{Retrieving the repository}
In order to clone a repository, it is important to confirm that Git is installed on the device. Following this, proceed to the repository's page on GitLab or GitHub and locate the "Clone" button. A URL to copy will be generated by clicking on this button. Next, launch the terminal or command prompt, navigate to the directory where the repository should be stored, and execute the command 'git clone [URL]', replacing [URL] with the URL you copied earlier. Doing this will generate a local copy of the repository on the device. If cloning is not possible, an alternative is to download the repository as a ZIP file. On the repository's page, look for the "Download ZIP" option, typically found in the same section as the clone option. Clicking on this will download the repository's contents in a compressed file, which can then be extracted to the preferred location.

\subsection{Simulating the environment}
The initiation of simulation files is the first step in simulating an environment, which requires the retrieval of specific files necessary for the simulated environment to function effectively. To achieve this, the cloned repository should be opened in the console. The submodule can be navigated by using the command.

\begin{verbatim}
$ cd. \src\main\resources\bthesis-provenancechain-digpat  
\end{verbatim}

Once the submodule is reached, the command 

\begin{verbatim}
$ git submodule foreach git fetch –tags
\end{verbatim}

should be executed. After it finishes, there should be no output. Finally, the command

\begin{verbatim}
$ git submodule update --init --recursive  
\end{verbatim}

should be run to conclude the process. This will ensure that the bthesis-provenancechain-digpat submodule contains the required .provn files.

\section{Building}
The jar file is packaged using the Maven Shade plugin, which is the preferred method. To create the jar file, navigate to the cloned repository using the console and run the \textbf{\texttt{'mvn clean package'}} command. After creating the jar file, it can be launched by executing the command \textbf{\texttt{'java -jar }\path{.\target\BThesis-ProvenanceChain-VERSION-shaded.jar'}}. In the event that the intended environment does not have a JRE, the \textbf{\texttt{'jpackage'}} command line tool provided by Java can be used to create a platform-specific installer.

\section{Omitting the simulated environment}
To facilitate traversal simulation, the implementation employs several classes and files that provide the necessary objects for the algorithm to function. These classes have been transferred to packages labeled 'bthesis.provenancechain.simulation' and 'bthesis.metageneration' to enhance clarity. At the same time, the required files reside in the previously mentioned submodule 'src.main.resources.bthesis-provenancechain-digpat'. However, these components can be omitted if the required classes are adequately substituted.
\shorthandon{-}


\chapter*{Conclusion}
\markright{\textsc{Conclusion}}
\addcontentsline{toc}{chapter}{Conclusion}
\shorthandoff{-}
Lorem ipsum dolor sit amet, consectetur adipiscing elit. Integer finibus commodo leo. Nullam blandit imperdiet magna, sit amet tempor tortor sagittis vitae. Lorem ipsum dolor sit amet, consectetur adipiscing elit. Cras elementum diam vel eros tristique, at maximus tortor ultricies. Curabitur urna magna, dictum at porta rutrum, congue nec justo. Nam ac rhoncus lectus. Ut feugiat volutpat ornare. Mauris quis neque nec lorem vestibulum iaculis. Proin posuere nisi eget nisi tristique, eu tempus nibh ultricies.
\shorthandon{-}

After linking a bibliography data\-base files to the document using
the \verb"\"\texttt{thesis\discretionary{-}{}{}setup\{bib\discretionary{=}{=}{=}%
\{\textit{file1},\textit{file2},\,\ldots\,\}\}} command, you can
start citing the entries. This is just dummy text
\parencite{borgman03} lightly sprinkled with citations
\parencite[p.~123]{greenberg98}. Several sources can be cited at
once: \cite{borgman03,greenberg98,thanh01}.
\citetitle{greenberg98} was written by \citeauthor{greenberg98} in
\citeyear{greenberg98}. We can also produce \textcite{greenberg98}%
\ or %% Let us define a compound command:
\def\citeauthoryear#1{(\textcite{#1},~\citeyear{#1})}%
\citeauthoryear{greenberg98}%
. The full bibliographic citation is:
\emph{\fullcite{greenberg98}}. We can easily insert a bibliographic
citation into the footnote\footfullcite{greenberg98}.

\printbibliography[heading=bibintoc]

\end{document}
