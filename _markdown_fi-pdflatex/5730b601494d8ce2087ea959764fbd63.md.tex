\markdownRendererDocumentBegin
\markdownRendererSectionBegin
\markdownRendererSectionBegin
\markdownRendererHeadingTwo{Set-up}\markdownRendererInterblockSeparator
{}\markdownRendererSectionBegin
\markdownRendererHeadingThree{Clone the repo}\markdownRendererInterblockSeparator
{}\markdownRendererOlBeginTight
\markdownRendererOlItemWithNumber{1}Clone/Code\markdownRendererOlItemEnd 
\markdownRendererOlItemWithNumber{2}Clone with HTTPS\markdownRendererOlItemEnd 
\markdownRendererOlItemWithNumber{3}In the console, run \markdownRendererCodeSpan{git clone <copied\markdownRendererUnderscore{}url>}\markdownRendererOlItemEnd 
\markdownRendererOlEndTight \markdownRendererInterblockSeparator
{}
\markdownRendererSectionEnd \markdownRendererSectionBegin
\markdownRendererHeadingThree{Init simulation files}\markdownRendererInterblockSeparator
{}In order for the simulated environment to run, you need to pull the required files 1. Open the cloned repo in the console 2. Move to the submodule using \markdownRendererCodeSpan{cd .\markdownRendererBackslash{}src\markdownRendererBackslash{}main\markdownRendererBackslash{}resources\markdownRendererBackslash{}bthesis-provenancechain-digpat} 3. Run \markdownRendererCodeSpan{git submodule foreach git fetch --tags} (There should be no output)\markdownRendererInterblockSeparator
{}\markdownRendererOlBeginTight
\markdownRendererOlItemWithNumber{4}Finish by running \markdownRendererCodeSpan{git submodule update --init --recursive}\markdownRendererOlItemEnd 
\markdownRendererOlItemWithNumber{5}The \markdownRendererCodeSpan{bthesis-provenancechain-digpat} submodule should now be populated with \markdownRendererCodeSpan{.provn} files\markdownRendererOlItemEnd 
\markdownRendererOlEndTight \markdownRendererInterblockSeparator
{}
\markdownRendererSectionEnd 
\markdownRendererSectionEnd \markdownRendererSectionBegin
\markdownRendererHeadingTwo{Building}\markdownRendererInterblockSeparator
{}The implementation uses a Maven Shade plugin for jar packaging\markdownRendererInterblockSeparator
{}\markdownRendererOlBeginTight
\markdownRendererOlItemWithNumber{1}Open the cloned repo in the console\markdownRendererOlItemEnd 
\markdownRendererOlItemWithNumber{2}Run \markdownRendererCodeSpan{mvn clean package}\markdownRendererOlItemEnd 
\markdownRendererOlItemWithNumber{3}Execute the created jar with \markdownRendererCodeSpan{java -jar .\markdownRendererBackslash{}target\markdownRendererBackslash{}BThesis-ProvenanceChain-VERSION-shaded.jar}\markdownRendererInterblockSeparator
{}\markdownRendererUlBeginTight
\markdownRendererUlItem alternatively, if the targeted environment doesn't have a JRE, you can create an exe installer using Java's prepackaged \markdownRendererCodeSpan{jpackage} command line tool\markdownRendererUlItemEnd 
\markdownRendererUlEndTight \markdownRendererOlItemEnd 
\markdownRendererOlEndTight \markdownRendererInterblockSeparator
{}\newpage \shorthandoff{-}\markdownRendererInterblockSeparator
{}
\markdownRendererSectionEnd \markdownRendererSectionBegin
\markdownRendererHeadingTwo{Omitting the simulated environment}\markdownRendererInterblockSeparator
{}In order to simulate the traversal, the implementation uses multiple classes and files to provide the required objects for the traversing algorithm to work. For clarity, these classes are moved to packages \markdownRendererCodeSpan{bthesis.provenancechain.simulation} and \markdownRendererCodeSpan{bthesis.metageneration}, while the required files are located in the submodule \markdownRendererCodeSpan{.\markdownRendererBackslash{}src\markdownRendererBackslash{}main\markdownRendererBackslash{}resources\markdownRendererBackslash{}bthesis-provenancechain-digpat} mentioned before. All of these can be removed as long as the classes needed are sufficiently replaced. \shorthandon{-}
\markdownRendererSectionEnd 
\markdownRendererSectionEnd \markdownRendererDocumentEnd